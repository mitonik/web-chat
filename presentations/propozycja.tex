\documentclass[aspectratio=169]{beamer}
\pdfsuppressptexinfo=1
\usepackage{polski}
\usepackage{hyperref}
\hypersetup{unicode,pdfdisplaydoctitle,
  pdftitle={Komunikator internetowy -- propozycja projektu},
  pdfauthor={Michał Antonik},
  pdfsubject={Indywidualny projekt programistyczny},
  pdfkeywords={komunikator internetowy, propozycja}
}
\usepackage{qrcode}
\beamertemplatenavigationsymbolsempty
\title{Komunikator internetowy}
\subtitle{Propozycja projektu}
\author{Michał Antonik}
\date{25 marca 2021}
\usetheme{Hannover}
\begin{document}
\maketitle
\section{Założenia}
\begin{frame}
  \frametitle{Założenia}
  Aplikacja internetowa służąca do komunikacji między połączonymi użytkownikami.
\end{frame}
\section{Funkcje}
\begin{frame}
  \frametitle{Funkcje aplikacji}
  \begin{itemize}
    \item Komunikacja w czasie rzeczywistym.
    \item Tworzenie pokoi z losowo generowanym identyfikatorem.
    \item Dołączanie przy użyciu linku.
    \item Wyświetlanie listy z nazwami użytkowników.
  \end{itemize}
\end{frame}
\section{Narzędzia i technologie}
\begin{frame}
  \frametitle{Narzędzia i technologie}
  \begin{itemize}
    \item Narzędzia
    \begin{itemize}
      \item Vue.js
      \item Node.js/Deno
      \item Socket.IO
    \end{itemize}
    \item Technologie
    \begin{itemize}
      \item HTTPS
      \item WebSocket
      \item UUID -- Progressive web application
      \item PWA -- Universally unique identifier
    \end{itemize}
  \end{itemize}
\end{frame}
\section{Harmonogram}
\begin{frame}[allowframebreaks]
  \frametitle{Harmonogram}
  2021-03-21 -- 2021-04-01
  \begin{itemize}
    \item Założenie i wstępne ustawienia projektu na Githubie.
  \end{itemize}
  2021-04-01 -- 2021-05-01
  \begin{itemize}
    \item Stworzenie wyglądu strony z podstawowym ułożeniem elementów.
    \item Zapoznanie się z narzędziami
    \begin{itemize}
      \item Vue.js
      \begin{itemize}
        \item Komponenty, zdarzenia, atrybuty.
        \item Vue CLI.
        \item Pobieranie danych.
      \end{itemize}
      \item Node.js/Deno
      \begin{itemize}
        \item Prosty serwer HTTP.
        \item Generacja identyfikatorów pokoi.
        \item Komunikacja dwukierunkowa między serwerem a klientami.
      \end{itemize}
      \item Socket.IO
      \begin{itemize}
        \item Zapoznanie się z API serwera i klienta.
      \end{itemize}
    \end{itemize}
  \end{itemize}
  2021-05-01 -- 2021-06-01
  \begin{itemize}
    \item Wyświetlanie listy użytkowników.
    \item Aktywność osób połączonych z czatem.
    \item Dopisywanie daty/godziny wysłania wiadomości.
  \end{itemize}
  2021-06-01 -- 2021-06-17
  \begin{itemize}
    \item Testy.
    \item Naprawianie błędów.
    \item Optymalizacje.
  \end{itemize}
\end{frame}
\section{Rozwinięcie projektu}
\begin{frame}
  \frametitle{Rozwinięcie projektu}
  \begin{itemize}
    \item Przesyłanie video/audio przy użyciu WebRTC.
  \end{itemize}
\end{frame}
\section{Kod}
\begin{frame}
  \frametitle{Kod projektu}
  \begin{columns}
    \begin{column}{.6\linewidth}
      \begin{center}
        \url{https://github.com/mitonik/web-chat}
      \end{center}
    \end{column}
    \begin{column}{.4\linewidth}
      \begin{center}
        \qrcode{https://github.com/mitonik/web-chat}
      \end{center}
    \end{column}
  \end{columns}
\end{frame}
\end{document}
